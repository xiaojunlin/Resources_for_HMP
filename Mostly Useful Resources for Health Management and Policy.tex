\documentclass[11pt, a4paper]{article}
\usepackage{ctex}
\usepackage{textcomp}
%行距
\linespread{1.5}
%首行缩进
\usepackage{indentfirst}
\setlength{\parindent}{2em}
%页边距
\usepackage[margin=1in]{geometry}
%图片
\usepackage{graphicx}
%表格
\usepackage{booktabs}
\usepackage{multirow}
\usepackage{tabularx}
\usepackage{threeparttable}
\usepackage{adjustbox}
\usepackage{enumerate}
%标题在表格上方
\usepackage{float}
\floatstyle{plaintop}
\restylefloat{table}
%链接
\usepackage{hyperref}
\hypersetup{colorlinks,linkcolor={blue},citecolor={blue},urlcolor={red}} 
\usepackage{xcolor}
%作者信息
\usepackage{authblk}
\title{\textbf{卫生管理系研究生可用资源}}
\author[1]{林小军}
\author[2]{蔡苗}
\affil[1]{华中科技大学同济医学院医药卫生管理学院卫生管理系 Email: \href{mailto:xjlin@hust.edu.cn}{xjlin@hust.edu.cn}}
\affil[2]{美国圣路易斯大学社会公正与公共卫生学院流行病与生物统计学系 Email: \href{mailto:miao.cai@slu.edu}{miao.cai@slu.edu}}
\date{\today}
\setcounter{Maxaffil}{0}
\renewcommand*{\Authfont}{\small}
\renewcommand\Authand{,  }
\renewcommand\Affilfont{\itshape\small}
\setlength{\affilsep}{2em}

%------------------------------------------------------------------------------
\begin{document}
\maketitle



%摘要
\begin{abstract}
这份资料是2014–2016年间,我们在同济医学院医药卫生管理学院硕士求学期间共同总结的一份学习资料,主要目的是帮助同门的师弟师妹更好的了解本学科的特点和发展趋势,为今后的研究生学习提供参考。两年后,我们重新整理了一遍文章内容,并结合自身的研究经验加入了一些新的内容,希望这份资料对大家有所帮助。

本文的结构如下:首先,我们总结了我们读过的一些关于中国医改问题比较经典的文献,推荐给大家阅读;然后,根据自己粗浅的研究经验,我们梳理了卫生管理研究的四个常见的研究方向,为大家今后选择自己的研究方向提供参考。我们也推荐了几个国内外的学术会议,加强同行之间的学术交流;随后,我们推荐了一些卫生服务研究和计量经济学相关的书籍,前者可以帮助大家更加深入地了解我国的医疗卫生服务体系,寻找研究方向和提出研究问题,后者则从计量经济学的角度为大家提供了解决研究问题的方法和工具;最后,我们针对大家比较关心的实践应用环节,总结了一些有用的学习资源,包括R、Stata等研究常用软件的学习资源,常用的科研小工具以及公开数据库资源等,方便大家在研究实践过程中快速上手。

前路漫漫其修远兮,吾将上下而求索。共勉!\\


\end{abstract}

\newpage
\tableofcontents


\newpage
\section{经典综述推荐}

\begin{enumerate}[(1)]
	\item Blumenthal D, Hsiao W. \textbf{Privatization and its discontents—the evolving Chinese health care system}[J]. New England Journal of Medicine, 2005, 353(11): 1165-1170.
	\item Hsiao W C. The political economy of Chinese health reform[J]. Health Economics, Policy and Law, 2007, 2(03): 241-249.
	\item Yip W, Hsiao W C. The Chinese health system at a crossroads[J]. Health Affairs, 2008, 27(2): 460-468.
	\item Yip W C M, Hsiao W C, Chen W, et al. \textbf{Early appraisal of China's huge and complex health-care reforms}[J]. The Lancet, 2012, 379(9818): 833-842.
	\item Hsiao W C. Correcting past health policy mistakes[J]. Daedalus, 2014, 143(2): 53-68.
	\item Yip W, Hsiao W. \textbf{Harnessing the privatisation of China's fragmented health-care delivery}[J]. The Lancet, 2014, 384(9945): 805-818.
	\item Yip W, Hsiao W C. What drove the cycles of Chinese health system reforms?[J]. Health Systems \& Reform, 2015, 1(1): 52-61.
	\item Meng Q, Fang H, Liu X, et al. \textbf{Consolidating the social health insurance schemes in China: towards an equitable and efficient health system}[J]. The Lancet, 2015, 386(10002): 1484-1492.
	\item He A J, Meng Q. An interim interdisciplinary evaluation of China’s national health care reform: emerging evidence and new perspectives[J]. Journal of Asian Public Policy, 2015, 8(1): 1-18.
	\item Li L, Fu H. China's health care system reform: Progress and prospects[J]. The International Journal of Health Planning and Management, 2017, 32(3): 240-253.
	\item Li X, Lu J, Hu S, et al. \textbf{The primary health-care system in China}[J]. The Lancet, 2017, 390(10112): 2584-2594.
	\item Liu G G, Vortherms S A, Hong X. \textbf{China's health reform update}[J]. Annual Review of Public Health, 2017, 38: 431-448.
\end{enumerate}

以上经典的研究综述,主要对中国近年来的医药卫生体制改革的问题剖析,是了解我国当前的医疗卫生体系很好的材料,基本上谈到中国的医改问题都会引用这些文章。如果觉得时间很少或者英文功底不强,请看黑体字标出来的这几篇文章,另外柳叶刀发表的几篇综述好像有中文翻译的版本。

William Hisao(萧庆伦)是哈佛大学公共卫生学院的教授,曾经也是保险领域的精算师,他的博士论文关于RBRVS方面的研究为整个美国的医师薪酬体系设计了一个全新的体系。

Winnie Yip(叶志敏)是牛津大学Blavatink政府学院的教授,也是哈佛大学的联合教授,研究侧重于卫生经济和卫生保险方面,蔡苗的文章曾经就被她拒过一次,感觉很荣幸。最近在宁夏搞了个医改试点,关于保险方面的,有兴趣的可以看看,跟伦敦热带医学院的教授Powel Jackon合作的几篇文章,后面这个教授来我们学院讲过课的。

William Hisao和Winnie Yip两人经常在大牛杂志上面发表关于中国医疗系统的综述类型的文章,两人也是中国政府医改的高级顾问,所以要说关于中国医改,恐怕没有比这两位更顶级和高层的专家了,两个人写的文章角度非常高,概括性很强,对于中国医改的问题也是一针见血。不过这些年不同文章的趋势看来,感觉两个人对于医疗体系的设计的看法也有点摇摆不定,有些文章里面把中国医疗体系的市场化说成是naïve,后面其他的文章又觉得这是个积极的尝试。

孟庆跃是北京大学公共卫生学院教授、院长,长期从事卫生政策与卫生经济研究,对中国的医改问题,特别是医疗保险制度改革有很深的研究见解。近年来在Lancet上发表多篇综述,高度概括了中国目前医改面临的问题和挑战。

刘国恩是中国卫生经济研究中心主任,Title很多,羌族的第一位博士,人生经历十分励志,有兴趣的可以了解一下。主要从事卫生经济学研究,研究水平非常高,对卫生经济问题的分析非常严谨深入。2018年,他中了一个国家自然科学基金重点项目《精准健康扶贫的经济学实验:基于四川凉山的干预研究》。建议对卫生经济研究有兴趣的同学可以关注一下刘国恩教授。

\section{研究方向}

\subsection{卫生政策与管理(Health Management \& Policy)}

\subsubsection{简述}

卫生政策与管理这个研究方向一直以来都是热门的研究方向,主要是从管理的角度,通过各种理论和方法对卫生政策进行分析、评估和改进,最终落脚点在于与卫生管理实际相结合,发挥卫生政策的最大效果。

中国卫生政策具有多样性且政策的实施环境复杂,政策设计和政策实施可能完全是两个样子,所以卫生政策制定者和国内外学者对于一项政策的实施效果还是非常关心的,特别是国家层面卫生政策。这个研究方向说好做也好做,说不好做也不好做。好做是因为中国这片土地太复杂了,有太多的卫生政策值得研究。每个地区都有其特异性,相应的卫生政策也有其独特性。比如上海是全国医改学习的榜样,然而上海的信息化好、钱又多、人群又是年轻人为主,这种环境下的卫生政策和经验有其独特性,在全国其他地区可能就难以复制或推广。不好做是因为卫生政策与管理研究需要有比较深的理论和方法研究,想要做好并不容易。以卫生政策评估为例,卫生政策评估是一个非常复杂的过程,其复杂性不仅体现在政策本身的复杂性,也体现在研究方法的复杂。比如国内很多中文文章对于卫生政策实施效果的评价就是简单的前后对比,然而这种前后对比其实只能反映政策实施前后变化,并不能反映政策实施的效果,二者有本质区别。真要去评估政策实施效果需要严格研究设计和方法学支持,比如随机试验、双重差分法、工具变量、断点回归设计等等。Abadie A, Cattaneo M D. Econometric methods for program evaluation[J]. Annual Review of Economics, 2018, 10: 465-503. 这篇文章梳理了卫生项目评估中常用的计量经济学方法,写的非常好,推荐对卫生政策评估有兴趣的同学细细研读。

目前中国研究的比较热门的卫生政策有医疗联合体(卫生服务整合)、城镇居民医保与新农合融合、药品零加成、薪酬制度改革等,这些都是比较具体的卫生政策,宏观一点的卫生政策就是公立医院改革、医保支付制度改革、供给侧改革、三医联动等医改相关内容了。如果你对这些政策感兴趣,而且擅长从管理的角度发现问题解决问题的话,可以考虑这个方向。

\subsubsection{代表性杂志}

\begin{enumerate}[(1)]
	\item \href{http://www.healthaffairs.org/}{Health Affairs} (HA)
	\item \href{http://www.healthpolicyjrnl.com/}{Health Policy}(HP)
	\item \href{http://www.valueinhealthjournal.com}{Value in Health} (VH)
	\item \href{https://academic.oup.com/intqhc}{International Journal for Quality in Health Care} (IJQHC)
	\item \href{http://heapol.oxfordjournals.org}{Health Policy and Planning} (HPP)
	\item \href{https://bmchealthservres.biomedcentral.com/}{BMC Health Services Research} (BMC HSR)
	\item \href{https://equityhealthj.biomedcentral.com/}{International Journal for Equity in Health} (IJEH)
	\item \href{http://www.sciencedirect.com/science/journal/02779536}{Social Science and Medicine} (SSM)
\end{enumerate}

关于卫生管理研究的杂志比较多,这里只列出了一些比较好的杂志。其中Health Affairs是卫生政策与管理研究的顶刊,值得大家好好关注,虽然关于中国的政策研究比较少,发上一篇的话,估计能当教授?HP顾名思义主要发表关于卫生政策研究的文章,但是该杂志发表比较多来自欧洲的研究,中国的研究非常少,我们学院目前可能就只有方鹏骞老师发过。VH实际上是药物经济学的旗舰,放在这里是因为药品的卫生政策也是一块很重要的研究内容,对药品政策感兴趣的可以关注。IJQHC这个杂志不像前面几个杂志,它没有明显的发表地区倾向,现在的主编是一个台湾人,杂志文章的水平比较高,属于我们努努力就可以够得着的杂志,我们学院有好几个老师都在上面发表过文章,投稿的时候可以考虑。HPP比较关注发展中国家的位卫生政策,对我们国家的卫生政策还是很感兴趣的。BMC HSR杂志收稿范围比较广,只要和卫生服务沾边的都可以投,杂志文章的整体水平一般,文章的发表周期比较长,从投稿到发表花个一年时间是常事,要投的话建议早点投。IJEH主要关注卫生服务的公平性,所以研究卫生服务公平性的文章可以往这里投,主编是约翰霍普金斯大学的石磊玉教授,他曾将来过我们学院作讲座。杂志整体水平挺高的,虽说发文量不小,但是也不是那么容易中的,杂志效率比较高,如果杂志不想要你的稿子,基本上3-5天就给你拒稿信了。SSM虽然不属于卫生管理大类,但是上面也不少卫生管理方面的文章,而且文章质量都比较高。

\subsubsection{典型文章}

\begin{enumerate}[(1)]
	\item Zhou B, Yang L, Sun Q, et al. Social health insurance and drug spending among cancer inpatients in China[J]. Health Affairs, 2008, 27(4): 1020-1027.
	\item Zhou M, Liu S, Kate Bundorf M, et al. Mortality In Rural China Declined As Health Insurance Coverage Increased, But No Evidence The Two Are Linked[J]. Health Affairs, 2017, 36(9): 1672-1678.
	\item Yi H, Miller G, Zhang L, et al. Intended and unintended consequences of China’s zero markup drug policy[J]. Health Affairs, 2015, 34(8): 1391-1398.
	\item Yip W, Powell-Jackson T, Chen W, et al. Capitation combined with pay-for-performance improves antibiotic prescribing practices in rural China[J]. Health affairs, 2014, 33(3): 502-510.
	\item Jian W, Lu M, Chan K Y, et al. Payment reform pilot in Beijing hospitals reduced expenditures and out-of-pocket payments per admission[J]. Health Affairs, 2015, 34(10): 1745-1752.	
\end{enumerate}

Zhou B等人的文章研究以癌症病人作为切入点,研究医疗保险与药品费用的关系。在2005–2006年期间,那时我国的医疗保险制度还不完善,很多病人是没有医保的,作者想知道癌症患者的药品费用差异和是否有医保是否存在关系,通过回归分析发现,有医保的癌症患者,他们的药品费用和没有医保的患者相比更高。在解释这种关系的时候,就要回到政策本身以及医患双方行为方面去,解释这种关系存在的可能原因,并提出相应的政策建议。文章关注的卫生政策的影响,选了一个大政策,但是研究的是一个很具体的费用影响,是典型的关联性研究文章。

Yip W等人在宁夏地区做了一个随机实验,研究按人头付费和按绩效付费结合的政策效果,是典型的政策效果评估文章。文章选择了基层机构医生抗生素处方情况,卫生费用,门诊服务量和患者满意度四个指标来测量政策效果。该研究采用了随机试验的研究设计,具有很强的证据性,比一般的观察性研究要强多了,但是这种研究很难找到地方做。


\subsection{卫生经济研究 (Health Economics)}

\subsubsection{简述}
卫生经济这个方向其实和卫生政策研究得比较紧密,不同的是,这个方向强调经济视角,注重研究假设和推断,在方法学方面要求比较高,常常涉及复杂的模型和方法。卫生经济是相当不推荐大家去做的一个方向,各种经济学模型、假设十分复杂,而且现在在卫生经济领域发表文章的学者往往具有强悍的经济学背景,文章晦涩难懂而且篇幅十分长(又臭又长)。经济领域普遍产量比较低,文章篇幅长可读性低,阅读群体少所以引用量低,也因而卫生经济类杂志影响普遍不高。最关键是出文章相当难。卫生经济方向最厉害的杂志\href{http://www.sciencedirect.com/science/journal/01676296}{Journal of Health Economics}, \href{http://onlinelibrary.wiley.com/journal/10.1002/(ISSN)1099-1050}{Health Economics},影响因子也就2~3分。在现在这个对产出和文章数量要求越来越高的时代,做这个方向性价比十分低。不过卫生经济方向的文章写的十分严谨,作者的经济计量功底都很深厚,文章里面经常会遇到各种数理推导,对模型的假设、适用性、稳健性都一一检定,从做研究的角度来说这类文章是很值得学习的。

\subsubsection{代表性杂志}
\begin{enumerate}[(1)]
	\item \href{https://academic.oup.com/qje/issue}{The Quarterly Journal of Economics} (QJE)
	\item \href{http://www.sciencedirect.com/science/journal/01676296}{Journal of Health Economics} (JHE)
	\item \href{http://onlinelibrary.wiley.com/journal/10.1002/(ISSN)1099-1050}{Health Economics} (HE)
	\item \href{https://www.sciencedirect.com/journal/china-economic-review}{China Economic Review} (CER)
	\item \href{https://www.cambridge.org/core/journals/health-economics-policy-and-law}{Health Economics, Policy and Law} (HEPL)
	\item \href{https://www.nber.org/}{The National Bureau of Economic Research}(NEBR)
\end{enumerate}

QJE是经济学研究的顶刊,虽然更卫生经济相关的文章很少,但是可以学习里面的研究思路和方法,一篇要读透要花很大功夫。JHE和HE两个杂志都是英国约克大学卫生经济研究中心办的,其中JHE是卫生经济研究的顶刊。两本杂志上面发表的文章水平都很不错,不过确实很少看到有中国人发表的文章。CER杂志里有很多来自中国的经济研究,时不时会发表卫生管理方面的文章,杂志水平不错。HEPL发文量不大,发文量不大,但是有不少大牛在上面发了文章,好像综述居多。NEBR不是一个杂志,它是美国国家经济研究所,里面有很多经济学者的工作论文,都是比较前沿的研究,可以参考学习里面关于卫生经济方面的文章。

\subsubsection{典型文章}

\begin{enumerate}[(1)]
	\item Pan J, Qin X, Li Q, et al. Does hospital competition improve health care delivery in China?[J]. China Economic Review, 2015, 33: 179-199.
	\item Pan J, Lei X, Liu G G. Health insurance and health status: exploring the causal effect from a policy intervention[J]. Health Economics, 2016, 25(11): 1389-1402.
	\item Qin X, Pan J, Liu G G. Does participating in health insurance benefit the migrant workers in China? An empirical investigation[J]. China Economic Review, 2014, 30: 263-278.
	\item Zhang Y, Salm M, van Soest A. The effect of retirement on healthcare utilization: Evidence from China[J]. Journal of Health Economics, 2018, 62: 165-177.
	\item Avdic D, Lundborg P, Vikström J. Estimating Returns to Hospital Volume: Evidence from Advanced Cancer Surgery[J]. Journal of Health Economics, 2018.
\end{enumerate}

这些都是卫生经济研究的典型文章,读这些文章的第一感觉就是怎么这么长。字那么小的情况下,都是十四五页往上走,加上附件的话,那会更长。第二感觉就是这么复杂的方法是怎么想出来的,看理论方法部分就已经可以把人弄晕过去了,可能看着看着就follow不上了。第三感觉就是,终于看完了,研究做的是真严谨,不得不服。举个例子来说吧,Pan J等人研究的中国医院竞争对于医疗服务提供的影响,文章首先就对中国的医疗服务市场的发展做了梳理,指出市场竞争的必要性,再分析了从内部竞争和外部竞争来分析市场竞争的内涵,文献综述分析了国内外学者对这个问题的研究结果,然后提出自己的研究假设。然后介绍复杂的研究方法和数据来源,利用了两个数据来从不同层面和不同角度分析医院竞争对医疗服务提供的影响。整体来看就是理论与方法结合的很紧密,一步一步求证,研究非常细致。说实在的,作为读者,要想一下子读明白这种卫生经济的论文是比较困难的,所以多花点时间去研究是有必要的。卫生经济研究者写出这种水平的文章一般都是要花比较长的时间,精雕细琢,而且杂志发表周期也很长,少则十个月,多则一年半载,所以能发出来都是心血之作。

\subsection{卫生产出研究(Health Outcome Research)}
\subsubsection{简述}
卫生产出研究其实是一个比较大的研究方向,说白了就是研究怎么提高卫生服务的产出结果。卫生产出结果有很多,从服务提供的角度看有质量、效率、费用、可及性等,从组织层面看有整个卫生体系、地区卫生服务、卫生服务机构、科室以及个人层面的产出,所以是这个研究方向可以研究的东西有很多。举几个例子来说,比如医疗服务质量,如何评价医疗服务质量?如何持续改进医疗服务质量?质量和效率、费用等其他方面怎么平衡?(Trade-off)这些都是很重要的问题,每个问题都可以做的很深。在评价医疗服务质量方面,我们经常使用的是死亡率,而且常常是住院死亡率。然而住院死亡率并不是一个很好的质量指标。首先,它只反映了住院期间的死亡,不能测量出院后病人的生存情况。因此国外常用的是30天死亡率。其次,对于某些死亡率较低的病种来说,探讨其死亡率的变化,实际意义不大,所以今后需要关注其他质量指标。

卫生产出研究和实际结合得非常紧密,而且基本上不搞什么复杂的模型研究和理论构建。但是这个方面的研究对数据质量的要求比较高,探讨卫生服务产出首先要准确测量产出结果,数据不准确谈什么都是白搭。另外,之前也提到了,很多产出指标已经做烂了,需要去找一些更有实际意义的指标,发挥对实际工作的指导价值。卫生产出研究中有很多关联性的研究,然而关联只是表象,后面潜在的机制才是关键,不然你怎么在实际工作中去改进呢?我发现我的身高和门口那棵树的高度相关,我为了长高你让我去把树拔高吗?这显然不是正确的做法。因此,卫生产出研究中的因果推断很受欢迎,可以关注一下。

\subsubsection{代表性杂志}
\begin{enumerate}[(1)]
	\item \href{http://qualitysafety.bmj.com/}{BMJ Quality and Safety} (BMJ Q\&S)
	\item \href{https://onlinelibrary.wiley.com/journal/14756773}{Health Services Research} (HSR)
	\item \href{https://journals.lww.com/lww-medicalcare/Pages/default.aspx}{Medical Care} (MC)
	\item \href{https://archpedi.jamanetwork.com/journals/jamanetworkopen}{JAMA Network Open} (JAMA Open)
\end{enumerate}

BMJ Quality \& Safety今年超越Health Affairs成为卫生管理研究的顶刊,比较关注如何在实际工作中提升病人的医疗服务质量和安全,值得大家好好关注,发表难度很高。HSR、Medical Care两本杂志比较关注美国的卫生问题,很少发表来自中国的研究,投稿中的希望不大。今年HSR的发文量增加不少,但是文章质量还是很高,一定程度上来说增大了大家的发表机会。JAMA Network Open这个杂志刚刚出来还没有影响因子,一般是第四年出前三年的影响因子,上面基本是来自牛津、哈佛、耶鲁等文章,阅读过好几篇,文章质量相当高,将来的影响因子不会低。不过版面费太TM贵了,有版面费报销的情况下可以试着投投,估计中的可能性不高。

\subsubsection{典型文章}
\begin{enumerate}[(1)]
	\item Birkmeyer J D, Siewers A E, Finlayson E V A, et al. Hospital volume and surgical mortality in the United States[J]. New England Journal of Medicine, 2002, 346(15): 1128-1137.
	\item Hentschker C, Mennicken R. The Volume–Outcome Relationship Revisited: Practice Indeed Makes Perfect[J]. Health services research, 2018, 53(1): 15-34.
	\item Bell C M, Redelmeier D A. Mortality among patients admitted to hospitals on weekends as compared with weekdays[J]. New England Journal of Medicine, 2001, 345(9): 663-668.
	\item Xu Y, Liu Y, Shu T, et al. Variations in the quality of care at large public hospitals in Beijing, China: a condition-based outcome approach[J]. PloS one, 2015, 10(10): e0138948.
	\item Hjelmar U, Bhatti Y, Petersen O H, et al. Public/private ownership and quality of care: Evidence from Danish nursing homes[J]. Social Science \& Medicine, 2018, 216: 41-49.
\end{enumerate}

第一篇是研美国究医院服务量和手术死亡率的关联性,但是没有回答这二者的因果关系。对于Volume-Outcome的关系,目前有两个经典的研究假设:(1)Practice-make-perfect hypothesis,即一开始医生的技能和经验都差不多,是不断的实践训练使得他们的技术和经验成长(learn by doing),高服务量给医生提升技艺提供了机会。(2)Selective-referral hypothesis,即医院的医疗产出越好,越有可能吸引病人前来就医或者被其他医院的医生转诊过来。而第二篇正是通过工具变量法来进行Volume-Oucome的因果推断。其他几篇都是研究医疗服务质量和某些因素的关联性,方法都比较简单。


\subsection{卫生服务效率研究(Efficiency Measurement in Health Care)}

\subsubsection{简述}

我们团队这个方向最开始做是程兆辉师兄,已经做到相当前沿的地步了,想要搞效率评价先把程师兄做过的东西看懂了再说。这个方向怎么说呢,说简单却是也简单,投入产出指标选好,拿个MaxDEA跑数据就完事了,管您三七二十一,上百种模型任君挑选,算出效率值就行。说难也难。首先,一般的医院效率评价已经被大家做烂了,指望靠BCC、CCR、两阶段、三阶段DEA等简单方法毕业估计已经比较难了,得想新的东西。其次,我们线性代数、微积分等知识可能不大好,很多方法的原理其实大部分都看不懂,知其然不知其所以然。

这个方向我觉得要么你在分析效率评价思路上有创新,要么你就要学会把效率评价和其他内容相结合。第一个方面,效率评价思路的创新,一般就是应用一下别人提出来的新方法,想要自己发明一种新算法估计很难,那是学数学、学统计的人做的事。说实在的,医疗领域的效率评价是比较落后于金融、能源等行业的,所以想要在方法方面有创新,你光看医疗领域的文章是没用的,得多看看其他领域的文章。DEA和SFA是效率评价的两大主流方法,拓展模型成百上千,推荐效率评价新手看看效率评价大牛Ozcan的书"Health Care Benchmarking and Performance Evaluation: An Assessment using Data Envelopment Analysis (DEA)",里面关于DEA模型及其应用讲的非常详细。稍微总结一下目前比较流行的、比较新的几种医院效率评价方法:Bootstrap DEA、SBM模型、方向距离函数、网络DEA、动态网络DEA、贝叶斯DEA以及SFA模型。需要注意的是,虽然DEA模型做效率评价的文章是主流,但是不要忘了SFA模型,它有它独特的优势,近年来SFA的文章也有不少,可以试试做SFA的研究。第二个方面也是最好出成果的方面,那就是不把重点放在效率评价上,而是将其与其他内容相结合,这就有非常多的东西可以做了。比如,程师兄把效率和医疗质量评价相结合,把风险调整死亡率融入效率评价,赋予效率值新的内涵;再比如,把效率和政策干预、市场竞争等相结合,探讨关联性。

\subsubsection{代表性杂志}
\begin{enumerate}[(1)]
	\item \href{https://www.journals.elsevier.com/european-journal-of-operational-research/}{European Journal of Operational Research} (EJOR)
	\item \href{https://www.journals.elsevier.com/omega}{Omega:The International Journal of Management Science} (Omega)
	\item \href{https://link.springer.com/journal/10729}{Health Care Management Science} (HCMS)
	\item \href{https://www.journals.elsevier.com/socio-economic-planning-sciences}{Socio-Economic Planning Sciences}(SEPS)
	\item \href{https://www.springer.com/public+health/journal/10916#}{Journal of Medical Systems} (JMS)
\end{enumerate}

EJOR和Omega都是运筹学的顶刊,上面有不少是医疗领域的效率评价的文章,想看前沿研究的话,看他们就对了。研究方法我觉得都比较复杂,看不懂直接跳过吧,觉得想用这个方法了再去找软件做。HCMS的主编是Ozcan,杂志影响因子虽然不高,但是文章水平都很高,主要都是医疗领域的效率评价。SEPS是一个老牌杂志,文章范围比较广,文章质量也比较高。JMS这个杂志发文量很大,很多效率评价的文章,曾因为自引比例过高被提出SCI目录,后来又回来了,分还不低。

\subsubsection{典型文章}
\begin{enumerate}[(1)]
	\item Cheng Z, Tao H, Cai M, et al. Technical efficiency and productivity of Chinese county hospitals: an exploratory study in Henan province, China[J]. BMJ open, 2015, 5(9): e007267.
	\item Cheng Z, Cai M, Tao H, et al. Efficiency and productivity measurement of rural township hospitals in China: a bootstrapping data envelopment analysis[J]. BMJ open, 2016, 6(11): e011911.
	\item Khushalani J, Ozcan Y A. Are hospitals producing quality care efficiently? An analysis using Dynamic Network Data Envelopment Analysis (DEA)[J]. Socio-Economic Planning Sciences, 2017, 60: 15-23.
	\item Hu H H, Qi Q, Yang C H. Analysis of hospital technical efficiency in China: Effect of health insurance reform[J]. China Economic Review, 2012, 23(4): 865-877.
	\item Narcı H Ö, Ozcan Y A, Şahin İ, et al. An examination of competition and efficiency for hospital industry in Turkey[J]. Health care management science, 2015, 18(4): 407-418.
	\item Lindlbauer I, Schreyögg J, Winter V. Changes in technical efficiency after quality management certification: A DEA approach using difference-in-difference estimation with genetic matching in the hospital industry[J]. European Journal of Operational Research, 2016, 250(3): 1026-1036.
\end{enumerate}
前两篇是程兆辉师兄的论文,第一篇是比较典型的DEA分析,目前是我们岛上被引最高的文章;第二篇是bootstrap-DEA在乡镇卫生院的应用;第三篇Ozcan的博士写的,用了网络DEA方法,这个方法现在用的人不多,关键是要找到一个好理解的网络模型,这比较难,这可能也是相关文章比较少的原因。我觉得目前这篇文章给出的网络模型是有问题的,虽然给出了有一定说服力的理由,但是缺乏理论支撑。曾经给这文章的第一作者发邮件,问她这么构建网络模型的原因和理论支撑,她说没啥理论支持,就是自己凭经验构建的。第四篇是将效率和医疗保险政策相结合,探讨医疗保险对医疗效率的影响,典型的两阶段DEA研究:第一阶段测效率,第二阶段分析影响因素;第五篇是探讨医院竞争对医院效率的影响;第六篇是研究质量管理认证对医院技术效率的影响,用了基因匹配发和DID相结合的方法,研究比较有意思,可以看看,开阔一下思路。


\subsection{小结和补充}
以上主要介绍了我们自己比较熟悉的四个研究方向,每个方向都有很多子方向,大家可以根据自己的兴趣去选择。我个人觉得研究方向的选择很重要,提前考虑好不是坏事,只是很少有人有这种想法,一方面老师在做的方向你得follow,另一方面自己未来未必就会接着搞研究。但提前想想研究方向的问题至少对于研究生毕业论文的选题是有帮助的,至少可以选一个自己感兴趣的做,而且有得选。对于研究方向的选择其实上面说的只能作为参考,喜欢做什么还是看个人,刚入门的时候多看看不同研究方向的文献,慢慢的就会知道自己的兴趣在哪里了。

目前来说,我们医管学院在各个研究方向都有老师在做,有时间可以和老师或者他们的学生聊聊,交流交流想法。目前学院的同行们发表文章较多的杂志是BMC Health Services Research、PLoS one、BMJ Open、Medicine、International Journal of Environmental Research and Public Health等,少数老师在Health Economics、Health Policy、International Journal for Quality in Health Care、Social Science and Medicine、Health Policy and Planning等水平较高的杂志上发表文章。

需要指出的是,除了英文杂志以外,国内也有一些不错的杂志可以关注的,比如:中国卫生政策研究、中国卫生经济、中国医院管理、中华医院管理等。同时也可以关注一些C刊,文章质量比较高,比如经济研究、管理世界、中国管理科学等,比只有两三页纸的中文期刊杂志要好上不少。

除了中国知网、万方、维普这三个常用的数据库以外,建议大家可以用用国家哲学社会科学文献中心(National Center for Pilosophy and Social Sciences Documentations, \href{http://www.ncpssd.org/about.aspx}{NCPSSD}),上面有很多C刊文献,是三大数据库上面搜不到的,关键是注册账号后下载免费,非常良心了。当你想换换脑子或者想看看CSSCI是怎么写卫生管理问题研究时,可以来看看,会有收获的。特别是,可以看看别的领域的人时怎么研究卫生问题的。举个例子来说,医疗行业的效率评价方法是远远落后于金融、能源等行业的,看看别人的文章有助于开阔研究视野,拓宽研究思路。


\section{学术会议推荐}
搞研究怎么能不和同行交流?怎么可以放过和大佬同框的机会?这几年,国内各大高校举办越来越多的学术会议,包括中山大学、四川大学、北京大学、
清华大学、上海交通大学等等,有些还是包食宿交通的,有机会可以去参加一下,多和领域内的同行交流,建立合作关系。有机会去做Presentation或者Speaker的话最好了,露露脸,也可以锻炼自己的表达能力,这非常重要。
\begin{table}[H]
\caption{卫生管理学术会议推荐}
	\centering
	\begin{threeparttable}
		\begin{tabular}{p{3cm}p{6cm}p{5cm}}
			\toprule
			会议名称 & 说明 & 网址 \\
			\midrule
			The Lancet-CAMS Health Summit & 柳叶刀和中国医学科学院合办的学术会议,自2015年起每年在北京举办一次,时间大约在10月底。大会收会议摘要,每年大概五月份左右就可以投稿了,不要错过。这可能是你离柳叶刀最近的一次。& \url{http://lancet-cams.medmeeting.org/6287?lang=en}\\
			华西卫生政策与经济博士生论坛 & 该论坛是由四川大学华西公共卫生学院、西部农村卫生发展研究中心主办的,至今已经办了三届了,口碑还不错,可以见见国内同行。& \url{http://wcsph.scu.edu.cn/info/1025/5146.htm}\\
			APHA Annual Meeting & 美国公共卫生协会的年会,规模很大,有上万人参会,应该是美国公共卫生领域最大的学术会议,美国很多学校的老师和老师都会参加。& \url{https://www.apha.org/events-and-meetings/annual/past-and-future-annual-meetings}\\			
			\bottomrule		
		\end{tabular}		
	\end{threeparttable}	
\end{table}





\section{书籍推荐}

\subsection{卫生服务研究相关书籍}
\begin{enumerate}[(1)]
	\item Burns L R., Liu G G. China's Healthcare System and Reform[M]. Cambridge University Press, 2017.
	\item 潘杰. 政府、市场与医疗[M].社会科学文献出版社, 2014.	
\end{enumerate}

第一本书是由美国Burns教授和刘国恩教授共同撰写的关于中国医药卫生体制改革的书籍,于2017年出版,里面提到了中美医疗卫生体系的比较、中国卫生体系的独特性,回顾了中国医改历程,分别从医院、医保和医药三个方面分析了中国医改的问题和对策。本书是英文版,美亚有售,国内暂未见到中文版发行。

第二本书是四川大学潘杰副教授的专著,主要是围绕“看病难、看病贵”的问题,对“市场机制”作用与“政府干预”措施进行经济学评估,比较深入地研究了中国医院市场竞争的问题,推荐对卫生经济评价感兴趣的同学阅读。

\subsection{计量经济学相关书籍}
\begin{enumerate}[(1)]
	\item Wooldridge J M. Introductory Econometrics: A Modern Approach[M]. Nelson Education, 2015.
	\item Wooldridge J M. Econometric Analysis of Cross Section and Panel Data[M]. MIT press, 2010.
	\item Angrist J D, Pischke J S. Mastering'metrics: The Path From Cause to Effect[M]. Princeton University Press, 2014.
	\item Angrist J D, Pischke J S, Pischke J S. Mostly Harmless Econometrics: An Empiricist's Companion[M]. Princeton: Princeton university press, 2009.
	\item James G, Witten D, Hastie T, et al. An Introduction to Statistical Learning[M]. New York: springer, 2013.
	\item 陈强. 高级计量经济学及Stata应用[M]. 高等教育出版社, 2014.
	\item 赵西亮. 基本有用的计量经济学[M].北京大学出版社, 2017.

\end{enumerate}

第一本书所有专业必读!从第一章到第九章,横截面数据的回归分析学完,你会对回归有一个完全不一样的了解,深入浅出的写出了回归的模型、假设、估计、推断等等,能够帮助你理解很多经济学上面的定义、回归模型为什么是这样的,用回归模型之前我要注意什么。第二部分的时间序列数据和面板数据其实讲的不是特别好,讲时间序列和面板的有很好的教材可以看。

第二本书是进阶版,建议谨慎购买,针对美国的研究生的教材,数理推导比较多,需要对线性代数和微积分基础比较好才能看的比较懂。不过对于一些进阶概念、面板数据,截断数据等有较为深入涉及。

第三本必读!是由两位经济计量大咖MIT的Angrist和LSE的Pischke写的一本对经济计量入门的一本书,里面对于数理推导比较少,从最基础的条件概率开始讲起,一步一步逐渐引入回归、工具变量、断点回归DID,相比下面一本语言通俗,数理推导很少,从最基础的内容开始引入。

第四本推荐喜欢经济方面的同学必读。还是上述两位大咖用生动鲜活的例子引入经济计量比较前沿的方法(现在已经不能算前沿),从选择性偏倚(Selection Bias)和随机试验讲起,逐渐引入因果关系与回归、工具变量、双重差分与面板数据、断点回归、分数位回归、聚集效应与序列相关。

第五本推荐喜欢统计学的同学阅读,里面对回归、高维数据的处理有非常详细的讲解,而且还有R命令帮助学习,需要一定的统计学基础。

第六本是山东大学的陈强教授写的一本书,从最简单的OLS到大样本OLS,到工具变量,到蒙特卡洛模拟与Bootstrap,到贝叶斯估计,基本上计量经济涉及到的问题这本书都囊括了,并且每一章后面都附有相应的stata操作指令,可以当做计量经济的“字典”用。

第七本是厦门大学经济学院赵西亮副教授写的一本书,主要从因果推断的基本思想出发,详细介绍Rubin潜在结果框架、随机化实验、匹配方法、回归方法、工具变量法、倍差法、断点回归法等现代经验分析方法,着重讨论了模型类型选择、模型变量选择、模型函数关系设定和模型变量性质设定的原则和方法。虽然里面有一些公式推导,但是不多,不懂没关系,不影响阅读。Stata命令代码不多,算是对第三本书的简化版。



\textbf{附:计量经济学值得学习的概念与模型}

\noindent{\textit{{必学概念与模型}}}
\begin{enumerate}[(1)]
	\item Ceteris Paribus
	\item Causal Effect/Causation 
	\item Endogeneity/Endogenous Variable 
	\item Confounding
	\item Selection Bias
	\item Simpson's Paradox 
	\item Log-transform in Regression(the Interpretations of Log(x), Log(y), and Both Log(x) \& Log(y)
	\item Heterogeneity 
	\item Bootstrap 
	\item Panel Data Models (Fixed-Effect/Random Effect Model, With-In Group Estimator, LSDV, Hausman Test) 
	\item Count Data Analysis(poisson Regression, Negative Binomial Regression, Overdispersion)
	\item Difference-in-Difference (DID) 
	\item Instrumental Variable (IV) 
	\item Propensity Score Matching (PSM)
	\item treatment Effect, Average Treatment Effect (ATE); Average Treatment Effect on the Treated (ATT); Average Treatment Effect on the Untreated (ATU)
\end{enumerate}


\noindent{\textit{{选学概念与模型}}}
\begin{enumerate}[(1)]
	\item Structural Equation Model (SEM)
	\item Regression Discontinuity Design (RDD)
	\item Data Envelopment Analysis (DEA) \& Stochastic Frontier Analysis (SFA)
\end{enumerate}


\section{软件学习资源}
数据科学其实是将来一个发展的趋势。医疗领域有海量的数据可以进行数据挖掘,懂得数据挖掘技术对医学、流行病和统计鲜有了解,拥有医学背景 的对数据缺乏敏感性,所以这个学科尤其需要有交叉学科的人才来进入探讨。 以我自己浅薄的见识看来,这个方向需要具有三个大的方面能力:编程、统计、医学。下面主要总结了一下编程方面的内容。

我记得在我读本科的时候,老师教的都是SPSS。操作相当傻瓜,点点点就好了,简单粗暴。那时候就知道这么一个统计分析软件,不知道天有多大,地有多广。后来慢慢接触到了R、Stata、SAS、Python等,才打开了新世界的大门。现在发展的越来越火的数据科学方面的几种编程语言:R, Stata, Python, SAS。对于R、Stata、SAS、Python等软件(或者叫编程语言更合适),我觉得并没有哪个比哪个好的问题,只有适合不适合你的问题,它们都有各自的优势和劣势,选择适合自己研究的才是最好的。我个人比较常用的是R和Stata。

\subsection{R}

R是近些年发展的特别火热的一种语言,也可以叫做一个软件,是由统计学家编写的一门编程语言。开源免费,软件包的内存也不大。其实严格意义上来说,它的语法风格不是像C++或者Java这种严格意义上编程语言需要繁琐的变量定义、类和对象设置,比较通俗易懂并且容易上手。里面的各种包特别特别丰富,大部分的统计学家都在使用R,所以最新的统计学家发明或者改良的算法和统计学模型,通常都会开发相对应的R包。一百分推荐给涉及到统计相关专业偏学院派的同学或者老师学习和使用。

现在随着RMarkdown等Rshiny等包的开发,我越来越觉得R是用来做学术的最优语言。Rmarkdown和Bookdown可以用来写学术文章和书籍(将文字、数据和R代码整合到一个文档中),去掉了在软件中学数据分析,然后复制粘贴到Excel进行结果整理,再复制到Word中写成文稿的繁琐过程。所有的步骤都在RStudio中完成,并且数据清洗、描述性统计分析、绘图和统计建模的步骤全部一清二楚,保证了研究的可重复性。Rmarkdown和Blogdown还可以用来开发网站。RShiny包的开发可以用来开发轻量级实现用户交互的网站,网站的前端和后端一起在R中实现,将来在商业和企业级用户中的前途可期。而且随着SparklyR, data.table等包的开发,R处理大数据的能力也在逐渐增强。总而言之,R现在很强,将来的潜力也很大,特别适合学术用途。

所以,作为卫生管理系的研究生(或者公共卫生),不用看后面的软件,你基本上也能通过以上描述判断出来自己应该学哪种语言。

\textbf{推荐使用R语言的软件:}

RStudio就不用多说了,一个用的最多的R集成开发环境(IDE),可以理解为更智能更好用的R。不用管IDE是什么,把资源区的两个文件下载安装了,然后每次打开RStudio就行。注意把R安装在磁盘的根目录下面,类似于D:\textbackslash R”,不要安装在Program Files下面,不然有一些从GitHub上面下载命令会出问题。


\textbf{推荐学习R的网站:}
\begin{enumerate}[(1)]
	\item John Hopkins的几个老师开设的用R学Data Science方面的课程:\\ \url{https://www.coursera.org/specializations/jhu-data-science}
	\item Duke大学的几个老师开设的用R学statistics的课程:\\ \url{https://www.coursera.org/specializations/statistics}
	\item Datacamp:可以学各种有用的R包 \\ \url{https://www.datacamp.com/}
	\item 中山大学《医学统计学》\\ \url{http://www.icourse163.org/course/sysu-20016#/info}
	\item Matthew Blackwell是哈佛大学政府学院一个助理教授,他的个人网站上面有很多他讲课的材料,包括Causual inference, quantitative methodology, controlling, missing data handling处理的一些 问题,有时间可以进去逛逛,页面很漂亮。 \\ \url{http://www.mattblackwell.org}
	\item Awesome R 上面推荐了很多好用的R包,总能找到一款你合适的R包\\ \url{https://awesome-r.com} 
	\item Google不用多说了吧 \\ \url{https://www.google.com}
\end{enumerate}

上面这几个资源真的是非常非常好的资源,John Hopkins应该是公共卫生世界排名top3的学校了,出的这门Data Science的课也是非常有价值,把你从对数据和R了解为0的小白领向数据科学家,不过学这门课花的时间也是会相当多的。总共包括十门课:数据科学家的工具箱,R编程,数据清洗,探索性数据分析(描述性),可重复性研究,统计推断,回归模型,机器学习,数据产品,数据科学capstone。课程设置循序渐进,只要你坚持学下来,绝对能够让自己的能力产生质变。有的人说这门课要收费,其实只要选financial aid然后填一些自己理由就可以看了,只不过拿不到结业证书而已。

Duke大学的课程则是从统计出发,课程假设你已经懂得一定的R编程知识,从概率论与数据,统计推断,线性回归和建模,贝叶斯统计和统计学Capstone五门课来把你逐渐带入统计学的殿堂。上课的老师是UCLA毕业的,用各种非常有趣的例子给你讲统计学,真的是现在看到的统计学讲的最好懂的老师了。她跟哈佛的一个博士后David M Diez合写几本书,就是这们课程的教材,是我在美国买到的最值的东西了。他们合作开设的一个讲统计学的网站\href{https://www.openintro.org/}{OpenIntro},是推广和教授统计学的网站,上面的资源真的是良心,跟着世界上最顶尖的老师从基础开始学统计,你们值得拥有。纸质版的书只要8刀,8刀买不了吃亏买不了上当,而且还是很厚的一本,在美国8刀吃最便宜的special lunch都不够的,简直残暴。什么,你连8刀都不想掏?他们连出版书pdf都挂在网上了,你可以自己下载下来看。不想自己敲书上的R代码?代码和相应的数据集也挂在网上了。不想看书?书的每一章的讲解也挂在网上了。视频都不想看?OK你无敌了。

其实学R的过程中教我学的最多的是谷歌,因为看啥书都没用,自己写代码的过程中还是会遇到各种稀奇古怪的报错,其实最后都得自己去谷歌。你会发现基本上你碰到的问题,几百年前就有人碰到过了,基本在stackoverflow上面都有人解答,一帮很热心的程序员。国内要翻墙要VPN才能上谷歌?试试一些Google镜像网站,不然就找个科学上网的姿势。

中山大学的公共卫生系是很厉害的,这门课一开始讲课的方积乾教授是我们之前上课的《卫生统计学》教材的主编,讲课让人印象深刻,主要是他那颗大痣。中文讲课,熟悉的灌输式上课风格,没事的时候可以2.0倍速度过一下,唤醒自己关于统计学的记忆。

\textbf{推荐学习R的教材:}
\begin{enumerate}[(1)]
	\item Kabacoff R I, 高涛, 肖楠, 等。 R语言实战[M]. 人民邮电出版社, 2013.
	\item 哈德利·威克姆,加勒特·格罗勒芒德, 陈光欣. R数据科学[M].人民邮电出版社, 2018.
	\item 赵鹏,李怡. 学R(零基础学习R语言)[M]. 研究出版社, 2018.
\end{enumerate}

第一本推荐用于R语言入门学习,从R的语法开始讲起,然后是数据处理,再是用R建立统计学模型,其实这本书都可以当做统计学教材了。实例很多,讲的也很细致。只不过本书最开始写与2008年,已经十年过去了,大清都亡了,所以很多函数都有更新更快的函数或者包替代了。

第二本是RStudio首席科学家Hadley Wickham和数据科学家Garrett Grolemund撰写的,包含了数据导入、清理、建模、结果输出等一整套工作流程,讲解详细且实用。原版在网上有免费的资源(\url{https://r4ds.had.co.nz}),翻译的版本难免有翻译不到位的地方,原版语言很简单,过了四六级基本上应该没有阅读障碍。

第三本作者赵鹏是Bookdownplus包的开发者,原书没有读过,不过看过他在网站发的很多帖子,干货十足并且语言风趣幽默,本书应该值得买。

\textbf{一定要学习的R包:}

推荐学习:tidyverse , dplyr, ggplot2, knitr, rmarkdown。前面三个都是大神Hadley Wickham写的包,knitr和rmarkdown是统计之都创始人谢益辉写的,两个现在都在R Studio工作。dplyr是用来对数据进行处理的,基本上excel上面能够进行的简单的数据处理操作都可以通过dplyr来做到,其实用基础的R命令也可以做到,不过据说dplyr处理大数据起来比较快。ggplot2是用R来进行画图的包,作图的思维很棒,用起来也很简单,做出来的图艺术气息也很浓,有一种说不出的美感,基本上用ggplot2做出来的图你一眼就能辨认出来。knitr是中国大神谢益辉写的,用来把R代码和输出结果优美的整合到一个文件上面去用的包,非常好用,学起来也特别简单,可以在这个上面使用Markdown的语法,最后可以把代码和每一步运算的数据结果输出到html5/pdf/MS word格式,方便代码和结果的相互传阅和使用。

其他常用的R包有:
\begin{table}[H]
\caption{其他比较常用的R包}
	\centering
	\begin{threeparttable}
		\begin{tabular}{lp{11cm}}
			\toprule
			R包 & 用途 \\
			\midrule
			devtools & 安装GitHub等上面的R包需要用到devtools包 \\
			rvest & 网络爬虫工具 \\
			MiaoCom & 计算合CCI、ECI、C3等合并症指数,由蔡苗开发维护,使用 devtools::install\_github("caimiao0714/MiaoCom@miao") 命令安装 \\
			outreg2 & 输出回归结果\\
			openxlsx & 用于打开和输出xlsx格式的Excel文件\\
			xtable & 表格输出工具\\
			shiny & 用于结果在网页上动态展示的工具,非常强大\\
			data.table & 处理大量的数据有优势,速度飞快,难度在于其语法相对难懂,学习门槛有点高\\
			RStata & R与Stata的交互工具,在R中可以调用Stata来运算\\
			broom & 回归结果整理工具\\
			haven & 读取和输出Stata,SAS等格式数据\\
			stargazer & 输出漂亮的回归结果,包括Latex和html\\
			sjstats & 常用的统计量计算\\
			kableExtra & 表格结果输出工具\\
			\bottomrule		
		\end{tabular}		
	\end{threeparttable}	
\end{table}


\subsection{Stata}

Stata是一个体积很小,功能很强大的软件。光help files都有3000多页,跟SAS有点像,用起来也很方便,可以像SPSS一样傻瓜式的点点点,也可以像R一样敲代码,下载外部命令,存成do.file。Stata可以实现的功能相当多,做经济计量和统计都可以用这个软件,不过其实使用这个软件的更多的是做经济的研究人员。有一些经济学期刊审稿的时候甚至会要求你上传源数据和Stata的do file,其严谨程度可见一斑。个人觉得学习的时候理论远远比Stata实践操作重要,Stata操作百度谷歌一下都可以知道,其背后的计量和统计理论才是真正有价值的东西。

\textbf{学习资源:}
\begin{enumerate}[(1)]
	\item 连玉君老师在YouKu上面的公开课:\\ \url{http://i.youku.com/i/UMzI3MjY4NzU4NA==?spm=a2h0k.8191405.0.0}
	\item 陈强老师的《高级计量经济学及Stata应用》。另外有条件的可以去听经管之家每年五一、十一、寒假都会举办的由连玉君和陈强老师的现场课。
	\item Stata资料大全,包括软件、视频、图书。\\ \url{http://180.76.139.143/pinggu/statashare.htm}
	\item UCLA的一个统计学网站, 里面包含了R、SAS、Stata、SPSS的统计学教程,里面有非常多的实例,是学习软件的好去处,谁用谁知道。\\ \url{http://www.ats.ucla.edu/stat/}
	\item LEMMA (Learning Environment for Multilevel Methods and Applications),学习多水平模型的网站,需要注册。网站提供讲义。软件包括R、Stata、SPSS、MLwin等。\\ \url{https://www.cmm.bris.ac.uk/lemma/}
\end{enumerate}

\textbf{Stata包推荐:}
Stata官方已经提供了非常多好用的命令,由于Stata是商业软件,上面发布的命令都比较权威,研究者可以放心使用。与此同时,和R类似,Stata上面也有很多研究者自己编写的外部命令,有些包还是非常好用的,但是外部命令的缺点是权威性不足,有可能出错。获取外部命令的最佳方式是使用 - findit - 命令,在搜索完成后,可以按照 stata 的提示直接下载安装相应的命令和作者提供的范例数据或dofiles(如果有的话)。输入 ssc hot, n(10) 可以呈现过去三个月关注度最高的 10 个命令,这些命令都可以使用 -ssc install- 命令直接安装到你的电脑里。

我目前觉得比较好用的Stata外部命令有以下几个:
\begin{table}[H]
\caption{常用的Stata外部程序包}
	\centering
	\begin{threeparttable}
		\begin{tabular}{lp{6cm}}
			\toprule
			程序包 & 用途 \\
			\midrule
			table1 & 生成数据的描述性统计表 \\
			esttab & 制作回归结果展示图表 \\
			estout & 输出回归结果 \\
			outreg2 & 输出回归结果\\
			carryforward & 填充缺失值\\
			\bottomrule		
		\end{tabular}		
	\end{threeparttable}	
\end{table}


\subsection{SAS}
SAS是老牌的做统计和数据科学的软件了(其实都不能称作是编程语言),正版软件处理数据的能力特别强大,毕竟是收费软件,各种技术人员给SAS开发命令,在美国这边做生物统计和数据科学应该是最权威的软件了。缺点就是桌面版软件超级大,占用硬盘空间比较多,另外由于是收费软件,所以需要自己科学开发使用。现在已经有免费的SAS University Edition出来了,需要在电脑上安装虚拟机使用,安装比较容易,而且感觉用户交互甚至比桌面版更好。在美国这边应聘比较古板的数据分析职业(比如FDA和药厂等,这些机构30多年前就已经开始用SAS,企业级用户想全面转变到其他平台太困难了),都会对SAS有要求。在一些功能的某些细节的实现上,SAS是其他软件无法替代的。

\subsection{Python}
Python也是近些年应用的特别火热的编程语言,在数据科学方面应用的也比较多,对数据进行处理和分析的能力也很强大,据说用来写抢票脚本很方便,不知道阿里那几个写程序抢月饼的程序员是不是用这个写的。跟R相比,Python更多的是应用在商业领域,如果要去公司做Data analyst,Python是必须要学的语言。如果是学院派,就在学校里面鼓捣几篇文章,其实R更适合你,这两个其实学一个就够了。Python学的不是很多,大家有想学的可以自行百度,网上海量资源。

\section{科研小工具}

\begin{enumerate}[(1)]
	\item SCI-HUB: 找不到全文的英文文献,老毛子开发的网站杠杠的,被无数美国公司起诉和告上国际法庭就是屹立不倒。(\url{http://sci-hub.tw}) SCI-HUB可用网址查询网址。(\url{https://lovescihub.wordpress.com})
	\item 微软的OneNote: 国内大部分人不用这个,但是这个就是个神器!
	\item Library Genesis: 下载电子书的神器,可以下载很多国外的原版教材,比如上面提到的“Introductory econometrics: A modern approach”, R语言实战英文原版。遇到想下载的英文书籍,不 妨试试,说不定有惊喜。(\url{http://gen.lib.rus.ec/search.php})
	\item \LaTeX: 排版精美,再也不用担心Word烦人的崩溃了,这份文档就是用\LaTeX 编写完成。
	\item GitHub:上面有很多代码和文档,你也可以将自己的代码和文档托管在GitHub上。上面的世界真的很精彩。
	\item Small PDF: 玩转PDF转换,谁用谁知道。(\url{https://smallpdf.com/cn}) 
	\item RearchGate:算是学者的个人学术档案,可以在上面发现和查询一些学者的研究成果,上心的学者一般会第一时间更新自己的研究成果。我一般用来订阅一些学者的动态和下载一些full-text,有兴趣不妨注册一个。(\url{https://www.researchgate.net})
	\item Google Scholar: 除了检索文献以外,它还是你的个人文献图书馆和学术档案,上面有你的学术发表记录,h指数等,也可以收藏文献、订阅学者的文章发表和被引情况、还会不定期给你推荐你可能感兴趣的文献。
	\item ORCiD:和Google scholar、ResearchGate有点像,也是个人的学术档案,只要你注册了,每个人都有唯一的ORCiD,有些杂志会把ORCiD作为投稿信息的一个可选项。(\url{https://orcid.org/})
	\item 世界银行报告文件库: 世界银行就不用多说了,发布的报告都很重量级。比如刚发布不久的“深化中国医药卫生体制改革:建设基于价值的优质服务提供体系(政策总论)”中英文双版本。(\url{http://documents.shihang.org/curated/zh/document-type})
	\item 美国国家学术出版社: 可以开放获取其出版物。内容丰富,有非常多的working paper和policy report在上面。(\url{http://www.nap.edu/})
	\item Web of Science和PubMed查询杂志名称简写(WOS: \url{https://images.webofknowledge.com/images/help/WOS/A_abrvjt.html}; PubMed: \url{https://www.ncbi.nlm.nih.gov/nlmcatalog/journals})
	\item Workflowy:用来建构提纲的小工具,也可以用来做读书笔记。(\url{https://workflowy.com/}) 
	\item Draw: 用来画一些流程图比较方便,是一个在线工具。如果你不想装Visio的话,它可以作为一个临时的替代品。(\url{https://www.draw.io})
	\item RSS:很多杂志都支持文章的RSS订阅功能,如果该期刊出了新的文章就会自动推送给你。我把上面比较好的杂志基本上都订阅了,每周看个几次,时刻关注最新的研究动态。
	\item ArcGIS: 如果对地理信息系统有兴趣,建议学习这个软件。
\end{enumerate}

我个人现在使用的Mac电脑,顺便推荐一下macOS生态下我比较常用的一些科研小工具:
\begin{table}[H]
\caption{macOS系统下的科研常用小工具}
	\centering
	\begin{threeparttable}
		\begin{tabular}{lp{10cm}}
			\toprule
			工具名字 & 用途 \\
			\midrule
			Reeder & 杂志RSS订阅,好看。 \\
			Papers 3 & 文献管理软件,数量不大的英文文献插入比较好使,文献很多或者有中文文献的时候不如EndNote好用。 \\
			Texpad & \LaTeX 工具,颜值高而且很方便。 \\
			PopClip & 每天都在用的快捷小工具,特别好用。举个例子,选中文章的doi号就可以在SCI-HUB上下载文献。\\
			PDF Expert 2 & PDF编辑工具,很好很强大,和Windows上的Adobe Acrobat DC差不多。\\
			Parallel Desktop & 虚拟机工具,用于安装Windows系统,以备不时之需。\\
			\bottomrule		
		\end{tabular}		
	\end{threeparttable}	
\end{table}




\section{公开数据库资源}
现在基于定量数据的实证研究越来越重要,可以说是“数据为王”了。很多idea没有数据的支撑很难做出来,下面整理了目前国内常见的公开的数据库,其中大部分数据都需要注册并申请才可以使用。

\begin{table}[H]
\caption{国内公开数据库资源}
\centering
\begin{adjustbox}{max width=\textwidth}
\begin{threeparttable}
\begin{tabular}{lp{9cm}}
	\toprule
	数据库名称 & 网址\\
	\midrule
	中国卫生与计划生育统计年鉴 & \url{http://www.moh.gov.cn/zwgkzt/pwstj/list.shtml}\\
	中国经济与社会发展统计数据库 & \url{http://tongji.cnki.net/kns55/Navi/NaviDefault.aspx}\\
	中国健康与养老追踪调查 & \url{http://charls.pku.edu.cn/zh-CN}\\
	国家统计局数据 & \url{http://data.stats.gov.cn}\\
	中国健康与营养调查 & \url{http://www.cpc.unc.edu/projects/china}\\
	中国综合社会调查 & \url{http://cnsda.ruc.edu.cn}\\
	中国家庭追踪调查 & \url{http://opendata.pku.edu.cn/dataverse/CFPS}\\
	中国高龄老人健康长寿调查 & \url{http://opendata.pku.edu.cn/dataverse/CHADS}\\
	中国家庭收入调查 & \url{http://ciid.bnu.edu.cn/chip/}\\
	中国劳动力动态调查 & \url{http://cnsda.ruc.edu.cn/index.php?r=projects/view&id=75023529}\\
	北京大学开放数据平台 & \url{http://opendata.pku.edu.cn}\\
	清华大学中国经济社会数据研究中心 & \url{http://166.111.5.104}\\
	中国家庭金融调查与研究中心 & \url{https://chfs.swufe.edu.cn}	\\
	\bottomrule
\end{tabular}
\end{threeparttable}
\end{adjustbox}
\end{table}



\section{微信公众号}
现在很多人都喜欢看一些微信公众号,确实也有不少好的资源,我推荐一些我经常看的微信公众号:
\subsection{计量经济学圈}
这个公众号主要是计量经济学爱好者的圈子,周围很多同学都关注了。每天都会有推送,而且推送的文章都比较有意思,质量还不错,干货很多。感兴趣的还可以加入他们的小组,需要付费,而且有分享的要求,但是关注这个公众号是不用付费的。(WeChat ID: econometrics666)

\subsection{计量经济学服务中心}
也是一个和计量经济相关的公众号,推文质量一般,有时会有一些有意思的推送。(WeChat ID: jingjixue100)

\subsection{华西卫生政策与经济}
四川华西卫生政策与政策研究方面的公众号,推文内容包括研究进展分享、讲座讯息和会议信息,很多同学都关注了这个。他们每年都会举办“华西卫生政策与经济博士生论坛”,可以关注一下。(WeChat ID: hxwszcyjj)

\subsection{患者安全}
主要是关于患者安全问题的一些推文,质量比较高,有些总结甚至比某些期刊文章写的还好。最近推送发的少了。(WeChat ID: patientsafety)

\subsection{烝民药经}
这是山东大学医药卫生管理学院左根永老师的公众号,我主要会看看他对一些中标基金的分析,有些讲的还比较好。(WeChat ID: PEHEOR)






\section{致谢}






\end{document}
